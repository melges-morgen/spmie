\chapter*{Заключение}						% Заголовок
\addcontentsline{toc}{chapter}{Заключение}	% Добавляем его в оглавление

В магистерской диссертационной работе сформулирована и решена важная в практическом отношении и интересной в теоретическом плане задачи оптимального расположения базовых станций широкополосной беспроводной сети вдоль протяженной транспортной магистрали. В настоящей работе получены следующие основные результаты:
\begin{enumerate}
	\item Дан обзор публикаций по тематике диссертации опубликованных в зарубежных статьях за последние годы.  На основе обзора сделан вывод о том, что предлагаемая в настоящей работе задача является новой, отличаясь от известных необходимостью учета ряда практических ограничений, которые в частности необходимо было реализовать при создании широкополосной беспроводной сети вдоль окружной дороги г. Казань (M7  ``Волга'')
	\item Во второй главе сформулирована общая задача оптимального расположения базовых станций в терминах целочисленного линейного программирования (ЦЛП). Дано точное численное решение указанной задачи с использованием пакета прикладных программ GLPK.
	\item Рассмотрен частный случай дискретной задачи с одним типом базовой станции, решение которой можно получить с полиномиальной сложностью.
	\item Для решения задачи небольшой размерности разработан алгоритм направленного перебора, обеспечивающий точное решение.
	\item Для исследования задач большой размерности предложен и исследован быстродействующий эвристический алгоритм, базирующийся на методе локальных улучшений. Проведен сравнительный анализ точного и приближенного решения. 
	\item Разработан комплекс программных средств для получения численных решений формулируемой задачи.
\end{enumerate}



\clearpage