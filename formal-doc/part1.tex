\chapter{Влияние ионосферы на прохождение радиосигналов}

Для описания и количественной оценки эффектов, возникающих при распространении электромагнитных волн в земной
атмосфере, необходимо знать её электрические свойства. В атмосфере Земли выделяются две существенно отличающиеся по
своим свойствам области: тропосфера (нижняя часть атмосферы, ниже 60км) и ионосфера (соответственно -- часть
атмосферы выше 60км) \cite{Kravcov}.

Тропосфера представляет из себя смесь нейтральных газов и её можно рассматривать как недиспергирующий изотропный
диэлектрик. В ионосфере же, существенную роль играет ионизация воздуха, обусловленная действием ультрафиолетового
излучения Солнца. Ионосферу в диапазоне УКВ\footnote{Ультракороткие волны -- радиоволны дапазона от метровых до
децимиллиметровых, т.е. от 30 МГц до 3000 МГц} волн можно рассматривать как холодную и анизотропную электронную
плазму \cite{Kravcov}.

\section{Строение ионосферы}

Ионосферой называют ионизированную область верхних слоев атмосферы Земли. Ионизация возникает главным образом под
действием ультрафиолетового излучения Солнца, в результате чего образуются положительно заряженные ионы и свободные
электроны. Кроме того, в процессе ионизации учавствуют рентгеновские лучи, излучаемые солнечной короной и
корпускулярные потоки Солнца. Вследствие низкой плотности атмосферы на большой высоте ионы и свободные электроны
рекомбинируют сравнительно медленно, образуется ионизированный слой газа, находящийся в состоянии динамического равновесия \cite{Kravcov}.

Ионосфера расположена на высотах от 60 до 1000км. Высота и плотность определяются интенсивностью 
УФ\footnote{Ультрафиолетовое} излучения, которое убывает по мере прохождения атмосферы, от разряженных её слоев к
более плотным. Ионосфера состоит из нескольких основных слоёв, плавно переходящих один в другой \cite{Kravcov}.

В дневные часы возникают четыре основных максимума ионизации:
\begin{itemize}
	\item Область $F_2$, высота 250-450 километров.
	\item Область $F_1$, высота 180-200 километров.
	\item Область $E$, высота 100-200 километров.
	\item Область $D$, высота 60-80 километров.
\end{itemize}

Слои $F_1$, $E$ и $D$ достаточно устойчивы \cite{Kravcov}.

После захода Солнца прекращается ионизация атмосферы и начинается процесс рекомбинации электронов, который более
активно проходит в плотных слоях атмосферы \cite{Kravcov}. Этим объясняется быстрое исчезновение наиболее 
низкого слоя D. Уменьшаются и сливаются слои $F_1$ и $F_2$. В ночные часы ионосфера состоит только из двух слоёв: 
слоя $E$, с пониженной концентрацией свободных электронов, и слоя $F_2$, который в ночные часы обозначается 
символом F без индекса.

По своим свойствам ионосфера эквивалентна полупроводнику \cite{Kravcov, NeQuick}. Поэтому ионосфере свойственны отражающие, преломляющие и ослабляющие свойства прохождению сквозь неё радиоволн.

Было установлено, что при частоте сигнала $F_{\text{кр}} = \sqrt{80,8 \cdot N_e}$ кГц, 
где $N_e$ -- удельная электронная концентрация в ионосфере, радиоволна перестаёт взаимодействовать с ионосферой. 
При $F > F_{\text{кр}}$ волна, пришедшая из космического пространства, преломляется и уходит к поверхности нашей планеты. 
А при $F < F_{text{кр}}$ волна отклоняется обратно или поглощается ионосферой \cite{Kravcov}.

Поглощающие свойства ионосферы зависят от колебаний свободных электронов. После возбуждения волной электрон
сталкивается с нейтральной молекулой или ионом газа и отдаёт полученную им энергию. Энергия радиоволны превращается
в энергию движения частиц газа, т.е. в тепловую. Исходя из этих постулатов, было произведена градация на радиоволны
КВ\footnote{Короткие волны, до 30мГц} и УКВ диапазонов. КВ диапазон используется для дальней радиосвязи между
объектами, находящимися на поверхности нашей планеты, УКВ для связи с ИСЗ\footnote{Искусственный спутник земли},
локации планет и радиоастрономических наблюдений.

\section{Модель ионосферы}

Характеристикой среды, в которой распространяются электромагнитные волны, служит тензор диэлектрической проницаемости $\hat{\varepsilon}(r, t)$ \cite{Kravcov}.

На частотах радиолокационного диапазона и в условиях статического поля Земли можно воспользоваться высокочастотным
приближением для тензора диэлектрической проницаемости холодной плазмы, в котором учтён только вклад электронов и
пренебрегается вклад ионов и влияние теплового движения. В данном приближении, компоненты тензора
\begin{equation*}
	\hat{\varepsilon} = 
		\begin{pmatrix}
			\varepsilon_{xx} & \varepsilon_{xy} & \varepsilon_{xz} \\
			\varepsilon_{yx} & \varepsilon_{yy} & \varepsilon_{yz} \\
			\varepsilon_{zx} & \varepsilon_{zy} & \varepsilon_{zz} \\
		\end{pmatrix}
\end{equation*}
определяющего линейную связь $D = \hat{\varepsilon} E$ между индукцией и напряжённо-
стью электрического поля, имеют вид:
\begin{equation*}
	\begin{aligned}
		&	\varepsilon_{xx} = 1 - \dfrac{v(1 + is)}{(1 + is)^2 - u}; \hfill \qquad &
			\varepsilon_{yy} = 1 - \dfrac{v[(1 + is)^2 - u\sin^2{\alpha}]}{(1 + is)[(1 + is)^2 - u]}; & \\
		&	\varepsilon_{yy} = -\varepsilon_{yx} = \dfrac{iv \sqrt{u} \cos{\alpha}}{(1 + is)^2 - u}; &
			\varepsilon_{yz} = \varepsilon_{zy} = 
				\dfrac{uv \cos{\alpha} \sin{\alpha}}{(1 + is)[(1 + is)^2 - u]}; & \\
		&	\varepsilon_{xz} = -\varepsilon_{zx} = -\dfrac{iv\sqrt{u}\sin{\alpha}}{(1 + is)^2 - u}; &
			\varepsilon_{zz} = 1 - \dfrac{v[(1 + is)^2 - u \cos^2{\alpha}]}{(1 + is)[(1 + is)^2 - u]}; &
	\end{aligned}
\end{equation*}
Здесь $\alpha$ -- угол между направлением распространения волны и статическим магнитным полем 
$H_0,v,u и s$ -- стандартные плазменные параметры:
\begin{equation*}
	v = \dfrac{{\omega_p}^2}{\omega^2}; \qquad
	u = \dfrac{{\omega_H}^2}{\omega^2}; \qquad
	s = \dfrac{v_{эф}}{\omega}
\end{equation*}
где ${\omega_p}^2 = 4\pi Ne^2/m$ -- квадрат плазменной частоты; $\omega_H = eH_0/mc$ -- гирочастота; 
$v_{эф}$ -- эффиктивная частота соударений электронов.

Численно значения параметров $v и \sqrt{u} = \dfrac{\omega_h}{\omega}$ удобно определять по формулам:
\begin{equation*}
	\begin{split}
		v = 3,18 \cdot 10^9 \cdot (\dfrac{N}{\omega^2}) = 8,06 \cdot 10^7 \cdot (\dfrac{N}{f^2}); \\
		\sqrt{u} = 0,22 \cdot 10^6 \cdot (\dfrac{H_0}{\omega}) = 3,52 \cdot 10^4 \cdot (\dfrac{H_0}{f})
	\end{split}
\end{equation*}
где $f = \dfrac{\omega}{2\pi}$ -- частота радиоволн в герцах; $H_0$ -- напряжённость магнитного поля, А/м. Для УКВ параметры $v, u и s$ малы по сравнению с единицей \cite{Kravcov}. Поэтому для моделей, связанных с УКВ, часто используются приближения основанные на малости этих параметров.

Пренебрегая всеми этими параметрами (в том числе влиянием геомагнитного поля), получаем, что тензор 
$\hat{\varepsilon}$ сводится к скаляру:
\begin{equation*}
	\begin{split}
		& \varepsilon_\omega (r, t) = 1 + \mu_\omega (r,t) \\
		& \mu_\omega(r,t) = -3,187 \cdot 10^3 \dfrac{N_e (r,t)}{\omega^2}
	\end{split}
\end{equation*}
$\mu_\omega(r,t)$ -- приведённая диэлектрическая проницаемость, $N_e$ -- электронная концентрация, 
$\omega = 2\pi f$, $f$ -- частота.

В условиях малости $\mu_\omega(r,t)$ влияние ионосферы на фазовый и групповой пути, углы прихода и доплеровское
смещение частоты электромагнитных волн описывается поправками первого порядка по приближениям 
геометрической оптики \cite{Kravcov, NeQuick, Gackovskiy}.

Поправка к групповому пути волны:
\begin{equation} \label{eq_distance}
	\begin{split}
		& D_l (R) = -\dfrac{1}{2} \int_{0}^{R} \mu(\vec{r_0}(s), t) ds = \dfrac{E(R)}{{\omega_c}^2} \\
		& E(R) = 1,593 \cdot 10^3 \int_{0}^R N_e(\vec{l_0}, t) ds \\
		& \vec{r_0} = \vec{l_0}s, \qquad 
		\vec{l_0} = \dfrac{\vec{R}}{R}, 
			\qquad R = |\vec{R}|, 
			\qquad \vec{R} = \vec{R}(t)
	\end{split}
\end{equation}
где $\vec{R} = \vec{R}(t)$ -- координаты объекта.

Поправка к фазовому пути:
\begin{equation*}
	L_l(R) = -D_l(R) = -\dfrac{E(R)}{\omega^2}
\end{equation*}
Поправка к доплеровской скорости:
\begin{equation*}
	V_l(R) = \dfrac{\partial{L_l}(R)}{\partial{t}} = \dfrac{1}{2}[\mu(\vec(R), t) V_r + \dfrac{V_t}{R}
		\int_{0}^{R} s \nabla \mu(\vec{r_0}(s), t) ds],
\end{equation*}
где $V_r$ -- радиальная скорость объекта, $V_t$ -- поперечная, по отношению к $l_0$, скорость объекта.
Поправка к единичному вектору направления прихода волны:
\begin{equation} \label{eq_angle}
	\vec{l_l}(R) = \dfrac{1,584 \cdot 10^3}{{\omega_c}^2} 
		\int_{0}^{R} (1 - \dfrac{s}{R} \nabla_{\perp} N_e(\vec{l_0}s) ds,
\end{equation}
где $\nabla_\perp \cdot \mu$ -- поперечный по отношению к $\vec{l_0}$ градиент функции $\mu(\vec{r})$