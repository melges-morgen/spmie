\chapter*{Введение}							% Заголовок
\addcontentsline{toc}{chapter}{Введение}	% Добавляем его в оглавление

{\setstretch{1.2}
РЛС наблюдают цели через околоземную среду, которую можно рассматривать как канал связи между РЛС и целями. 
Поэтому РЛС в большей или меньшей степени подвержены разнообразному влиянию околоземной среды. Характер и
величина этого влияния зависит от состояния околоземной среды, типа и параметров ее возмущений, от используемого
в средстве диапазона длин электромагнитных волн и координат РЛС.

Компоненты околоземной среды имеют различные электромагнитные характеристики. Для нейтральной и ионизированной
компонент среды основной электромагнитной характеристикой является комплексная диэлектрическая проницаемость
$\varepsilon$, которая определяет показатели преломления и поглощения
электромагнитных волн в зависимости от частоты. Для конденсированной компоненты, образованной облачными и
аэрозольными частицами, основными электромагнитными характеристиками являются объёмные поперечники рассеяния и
поглощения.

Для определения диэлектрической проницаемости нейтральной компоненты (атмосферы) необходимо задавать давление,
температуру, влажность, для ионизированной компоненты (ионосферы) -- плотность и температуру электронов, ионов и
нейтральных частиц, для конденсированной компоненты -- распределение частиц (облачных и аэрозольных) по составу
и размерам, полную концентрацию частиц.

Строго говоря, указанные параметры являются случайными функциями пространственных координат и времени. 
В нормальных условиях средние значения этих параметров зависят от географических координат (долготы и широты),
высоты, времени суток, сезона, а также специфических индексов активности влияния внешних факторов, к числу
которых относятся индексы солнечной, геомагнитной и вулканической активности. Параметры облачности и облачных
частиц зависят, к тому же, и от метеорологических условий. В настоящее время существуют достаточно надежные 
эмпирические модели пространственно-временных изменений средних значений параметров компонент околоземной среды
в естественных нормальных условиях, в частности, для ионизированной компоненты -- модель ионосферы
(IRI, Nequick).

Проведённые к настоящему времени исследования эффектов взаимодействия электромагнитных волн радиолокационных
диапазонов с пространственно неоднородной околоземной средой сделали возможным получение надежных
количественных моделей эффектов прохождения сигналов в околоземной среде. Важнейшей частью работы современных 
радиолокационных устройств является задача компенсации искажений вносимых ионосферой.

\textbf{Целью} данной работы является доработка программного иммитатора влияния околоземной среды, разработка и
реализация моделей восстановления орбиты КО\footnote{Космических Объектов} с учётом и без влияния околоземной 
среды, тестирование и сравнение работы алгоритмов реализующих модели.

% Описать структуру
}

\clearpage