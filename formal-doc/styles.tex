%%% Макет страницы %%%
% Выставляем значения полей (ГОСТ 7.0.11-2011, 5.3.7)
\geometry{a4paper,top=2cm,bottom=2cm,left=2.5cm,right=1cm}

%%% Кодировки и шрифты %%%
\ifxetex
  \setmainlanguage{russian}
  \setotherlanguage{english}
  \defaultfontfeatures{Ligatures=TeX,Mapping=tex-text}
  \setmainfont{Times New Roman}
  \newfontfamily\cyrillicfont{Times New Roman}
  \setsansfont{Arial}
  \newfontfamily\cyrillicfontsf{Arial}
  \setmonofont{Courier New}
  \newfontfamily\cyrillicfonttt{Courier New}
\else
  \IfFileExists{pscyr.sty}{\renewcommand{\rmdefault}{ftm}}{}
\fi

%%% Интервалы %%%
\linespread{1.3}                    % Полуторный интвервал (ГОСТ Р 7.0.11-2011, 5.3.6)

%%% Выравнивание и переносы %%%
\sloppy                             % Избавляемся от переполнений
\clubpenalty=10000                  % Запрещаем разрыв страницы после первой строки абзаца
\widowpenalty=10000                 % Запрещаем разрыв страницы после последней строки абзаца

%%% Библиография %%%
\makeatletter
\bibliographystyle{utf8gost71u}     % Оформляем библиографию по ГОСТ 7.1 (ГОСТ Р 7.0.11-2011, 5.6.7)
\renewcommand{\@biblabel}[1]{#1.}   % Заменяем библиографию с квадратных скобок на точку
\makeatother

%%% Изображения %%%
\graphicspath{{images/}}            % Пути к изображениям

%%% Цвета гиперссылок %%%
\definecolor{linkcolor}{rgb}{0, 0, 0} % {0, 0.125, 0.376} - ultramarine
\definecolor{citecolor}{rgb}{0,0.6,0}
\definecolor{urlcolor}{rgb}{0,0,1}
\definecolor{commentcolor}{gray}{0.5}
\hypersetup{
    colorlinks, linkcolor={linkcolor},
    citecolor={citecolor}, urlcolor={urlcolor}
}

%%% Оглавление %%%
\renewcommand{\cftchapdotsep}{\cftdotsep}

%%% Шаблон %%%
\newcommand{\todo}[1]{\textcolor{red}{#1}}

%%% Списки %%%
% Используем дефис для ненумерованных списков (ГОСТ 2.105-95, 4.1.7)
\renewcommand{\labelitemi}{\normalfont\bfseries{--}} 

%%% Колонтитулы %%%
% Порядковый номер страницы печатают на середине верхнего поля страницы (ГОСТ Р 7.0.11-2011, 5.3.8)
\makeatletter
\let\ps@plain\ps@fancy              % Подчиняем первые страницы каждой главы общим правилам
\makeatother
\pagestyle{fancy}                   % Меняем стиль оформления страниц
\fancyhf{}                          % Очищаем текущие значения
\fancyhead[C]{\thepage}             % Печатаем номер страницы на середине верхнего поля
\renewcommand{\headrulewidth}{0pt}  % Убираем разделительную линию

% Tikz style (Модуль рисования графов)
% Block scheme style
\tikzstyle{decision} = [diamond, draw, fill=blue!20, 
    text width=5.7em, text badly centered, node distance=3cm, inner sep=0pt]
\tikzstyle{block} = [rectangle, draw, align=center, fill=blue!20, minimum width=10em,
	text centered, rounded corners, minimum height=4em, node distance=3cm]
\tikzstyle{start} = [circle, draw, align=center, fill=green!20, minimum width=2em,
	text centered, rounded corners, minimum height=2em, node distance=4cm]
\tikzstyle{end} = [circle, draw, align=center, fill=red!20, minimum width=2em,
	text centered, rounded corners, minimum height=2em, node distance=4cm]
\tikzstyle{line} = [draw, -latex']

\tikzstyle{cloud} = [draw, align=center, ellipse,fill=red!20, node distance=3cm,
    minimum height=2em]
    
\tikzstyle{input} = [draw, align=center, ellipse,fill=yellow!20, node distance=3cm,
    minimum height=2em]
\tikzstyle{output} = [draw, align=center, ellipse,fill=red!20, node distance=4cm,
    minimum height=2em]

% TLE Format Diagram
\tikzstyle{tle} = [rectangle, draw, align=center, fill=white!20, minimum width=0.5cm,
	text centered, minimum height=0.5cm, node distance=0cm]
\tikzstyle{tle-green} = [rectangle, draw, align=center, fill=green!20, minimum width=0.5cm,
	text centered, minimum height=0.5cm, node distance=0cm]
\tikzstyle{tle-yellow} = [rectangle, draw, align=center, fill=yellow!20, minimum width=0.5cm,
	text centered, minimum height=0.5cm, node distance=0cm]