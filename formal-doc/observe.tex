\chapter{Обзор зарубежных публикаций по проблеме синтеза топологии беспроводных сетей}	\label{observation}
Учитывая высокие требования к безопасности, использование сетей общего пользования (типа Интернет) в системах связи вдоль протяженных трасс обычно не допускается, тем более, что протяженные магистрали часто проходят по малонаселенным, труднодоступным территориям, где доступ к сети Интернет или сотовой связи отсутствует ~\cite{Vishnevsky1, Vishnevsky2}. Создание выделенных оптоволоконных сетей вдоль протяженных магистралей или радиорелейных линий требует огромных материальных затрат. То же касается и использования спутниковых каналов и сетей связи. В то же время стоимость широкополосной высокоскоростной беспроводной связи на аппаратно-программных средствах, реализующих международный стандарт IEEE 802.11-2012 ~\cite{802.11-2012}, на порядок ниже. Указанный стандарт регламентирует создание высокоскоростных каналов связи и беспроводных сетей, функционирующих под управлением протоколов IEEE 802.11n и IEEE 802.11s, на базе которых могут эффективно реализовываться беспроводные сети вдоль протяженных транспортных магистралей. Указанные сети обеспечивают создание не только магистральной скоростной передачи мультимедийной информации путем расположения базовых станций на высотных зданиях и вышках вдоль транспортных магистралей, но и оперативную связь со стационарными и мобильными абонентами (автомобили, поезда, дорожные знаки, пункты весового контроля и пункты контроля ПДД, управления светофорами и т.д.). 
Широкополосные беспроводные сети и каналы связи стали в настоящее время одним из основных направлений развития телекоммуникационной индустрии. Этот факт нашел отражение в многочисленных зарубежных монографиях, статьях в специализированных журналах: IEEE Wireless Communications Magazine, IEEE Communications; IEEE Wireless Network; IEEE Transactions on Vehicular Technology и трудах конференций (IEEE GLOBECOM, IEEE INFOCOM, ICUMT, NetWare, DCCN и многих других), где исследовались архитектура, методы оценки и оптимизации параметров протоколов, математические модели расчета производительности и надежности  беспроводных сетей и каналов связи и т.д. Исследованию широкополосной связи посвящены и монографии отечественных авторов, например ~\cite{Vishnevsky1, Vishnevsky2} и отдельные статьи в журналах: Автоматика и телемеханика, Радиоэлектроника, Телекоммуникации и транспорт, Проблемы информатики, Электроника, Беспроводные технологии и т.д. 
Развертывание и развитие сетей беспроводной связи вдоль протяженных магистралей требует решения ряда сложных организационно-технических задач в условиях жестких ограничений на использование частотных, экономических и аппаратных ресурсов. В связи с этим возрастает актуальность решения проблемы оптимального размещения базовых станций вдоль транспортных магистралей, которая является одной из важнейших при проектировании широкополосных беспроводных сетей этого класса. Ее решение направлено как на реализацию высокоскоростной магистральной сети, так и максимальное телекоммуникационное покрытие трассы с целью обеспечения подключения мобильных пользователей, а также минимизации интерференции и временных задержек при передаче мультимедийной информации по сети.
Исследованию этой проблемы посвящены многочисленные отечественные и зарубежные публикации ~\cite{Conference1, Conference2, Liu, Reis, Lee1, Lee2, Xie, Wu}. В частности, в ~\cite{Conference1} решена задача размещения базовых станций беспроводной сети по критерию максимального покрытия протяженной трассы при ограничениях на суммарную стоимость сети. Исходными данными для решения задачи являются потенциальные места установки базовых станций, а также предварительно собранная статистика трафика от стационарных и мобильных абонентов. Близкая по постановке задача размещения базовых станций, максимизирующая зону покрытия, приведена в ~\cite{Conference2}. Для аналитического описания задачи используется модель максимального покрытия с ограничением на время задержки, для исследования которой предложен генетический алгоритм.
Стратегии размещения базовых станций, учитывающие параметры дорожного трафика и направленные на улучшение качества связи в автомобильных ad-hoc сетях, рассмотрены в работе ~\cite{Liu}. Для нахождения оптимальных зон покрытия каждой станции авторами предложен алгоритм расширения и раскраски. Проблема отыскания оптимальной стратегии сформулирована как комбинаторная оптимизационная задача максимизации вероятности связности путем поиска оптимального расположения базовых станций. В качестве примера проведен расчет на большом участке городской дорожной сети. Приводятся также результаты имитационного моделирования, показывающие, что схема расположения базовых станций, полученная при помощи описанной в статье стратегии, позволяет получить лучшую связность сети по сравнению с предложенными ранее методами.
Проблема размещения дорожных базовых станций в рамках сетей стандарта IEEE 802.11p/WAVE изучается в работе ~\cite{Liu}. Представлена аналитическая модель, позволяющая анализировать задержку передачи данных в сетях на автомагистралях. Исследованы случаи связанных и несвязанных базовых станций. Показано, что эффективными оказываются лишь те стратегии размещения, в которых дорожные базовые станции связаны друг с другом в пределах прямой видимости. В работе ~\cite{Lee1} описывается анализатор, позволяющий оценивать длительность периодов наличия и отсутствия соединений движущихся транспортных средств с дорожной базовой станцией, основываясь на предварительно собранных телематических данных, а также на топологии опорной дорожной сети. Предложен корректный выбор схемы расположения базовых станций, позволяющий значительно увеличить зону покрытия сети. 
Результаты расчетов и моделирования схемы расположения базовых станций вдоль дорог в городе Чеджу (Южная Корея) приводятся в статье ~\cite{Lee2}. Рассчитанная оптимальная топология опорной беспроводной сети обеспечила повышение надежности соединений и сокращение интервалов времени, в течение которых транспортные средства остаются без связи. В статье ~\cite{Xie} рассмотрена задача оптимального расположения станций, предназначенных для рассылки информационных сообщений движущимся по городским дорогам транспортным средствам. Предложена схема выбора расположения станций, не использующая сведений о собранных точных маршрутах движения транспортных средств. Проблема размещения базовых станций в сети транспортных средств, расположенных на автострадах или иных дорогах, содержащих большое количество полос со съездами или пересечениями на протяжении дороги, изучается в статье ~\cite{Wu}. В данной модели каждое транспортное средство может иметь связь с базовой станцией двумя способами: доставка информации напрямую, если транспортное средство оказывается в области прямой видимости с базовой станцией; многошаговая передача, происходящая в случае, если транспортное средство покидает пределы прямой видимости. В предлагаемой стратегии рассчитываются оба способа доступа и формулируется проблема размещения с использованием модели целочисленного линейного программирования таким образом, чтобы общая пропускная способность сети была максимальной. При формулировке задачи учитывается влияние интерференции, распределение потоков транспортных средств и их скорость. С использованием системы моделирования NS-2 проведена оценка основных характеристик беспроводной сети при применении предложенной стратегии.

В отличие от описанных выше задач размещения в настоящей работе рассматривается задача оптимального размещения базовых станций в дискретной постановке при наличие ряда практических ограничений, которые необходимо было реализовать при создании широкополосной беспроводной сети вдоль окружной дороги г. Казань (M7  ``Волга'').
Задача состоит в том, чтобы расставить базовые станции, отличающиеся дальностью радиорелейной связи, радиусом покрытия для абонентского доступа и ценой, так, чтобы покрыть максимально возможную часть дороги, при имеющихся ограничениях на бюджет.