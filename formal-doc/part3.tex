\chapter{Программная реализация} \label{ch:programm}
В предыдущей главе была составлена общая абстрактная модель имитации, в этой же главе будет описана её реализация.
Для написания программы был выбран язык C++, в качестве библиотеки графического интерфейса -- Qt.
Объектноориентированный стиль программы хорошо подошёл для поставленной задачи. 
Во-первых, в программе сами собой напрашиваются многие объекты: РЛС, спутник и пр., во-вторых, между ними хорошо
строятся наследственные связи (это будет показано далее по тексту), а в-третьих, объектно-ориентированный подход
позволил спроектировать программу таким образом, чтобы любой её модуль или даже часть модуля
можно было легко заменить или сделать несколько реализаций.

\section{Общая информация}

Программа разделена на четыре основные части:
\begin{itemize}
	\item Библиотека для иммитации и восстановления параметров орбиты по результатам наблюдений, содержащую в
			себе всю логику относящуюся к симулированию радионаблюдений, реализацию всех необходимых для этого
			моделей из предыдущей главы и прикладные модули, необходимые для
			нормальной работы.
	\item Библиотека для работы с методом градиентного спуска и методом наименьших квадратов.
	\item Графический интерфейс, позволяющий удобно использовать написанную программу.
	\item Текстовый постпроцессор, позволяющий выгружать данные для обработки в других программах.
\end{itemize}

Читая текст работы может создаться впечатление, что архитектура программы была спроектирована заранее (т.к.
изложение идёт от постановки задачи к её реализации модуль за модулем), но это совсем не так. Программа множество
раз меняла свою структуру, реализацию, архитектуру и пр., по мере улучшения понимания поставленной задачи и её
теоретических аспектов. Однако не все части программы переписывались, для идеального вписывания в новую архитектуру,
из-за этого некоторые функции могут не использоваться, классы могли получится перегруженными или наоборот --
слишком разряженными, API местами неудобным и т.д. Всё это результат многократного рефакторинга. 
Кончено, в будущем все такие недостатки можно исправить.
\newpage

\section{Парсинг TLE-файлов}

Для парсинга TLE-строк описанных в параграфе \ref{sbs:tle-module} на языке С++ можно воспользоваться функцией 
\textit{sscanf} (см. листинг \ref{lst:TLE-Read-String}), предварительно заменив все пробелы на нули на позициях,
предшествующих десятичным разделителям т.к. в TLE-формате принято не писать 0 перед десятичной частью.
\begin{lstlisting}[language=C++, basicstyle=\fontsize{10}{9}\ttfamily,
	caption={Чтение TLE строк},label={lst:TLE-Read-String}]
if (longstr1[44] != ' ')
	longstr1[43] = longstr1[44];
longstr1[44] = '.';
if (longstr1[7] == ' ')
longstr1[7] = 'U';
if (longstr1[9] == ' ')
	longstr1[9] = '.';
for (j = 45; j <= 49; j++)
	if (longstr1[j] == ' ')
		longstr1[j] = '0';
if (longstr1[51] == ' ')
	longstr1[51] = '0';
if (longstr1[53] != ' ')
	longstr1[52] = longstr1[53];
longstr1[53] = '.';
longstr2[25] = '.';
for (j = 26; j <= 32; j++)
	if (longstr2[j] == ' ')
		longstr2[j] = '0';
if (longstr1[62] == ' ')
	longstr1[62] = '0';
if (longstr1[68] == ' ')
	longstr1[68] = '0';
// First tle-line
sscanf(longstr1,"%2d %5ld %1c %10s %2d %12lf %11lf %7lf %2d %7lf %2d %2d %6ld ",
                 &cardnumb,&satrec.satnum,&classification, intldesg, 
                 &satrec.epochyr, &satrec.epochdays,&satrec.ndot, 
                 &satrec.nddot, &nexp, &satrec.bstar, &ibexp, &numb, &elnum );
// Second tle-line
if (longstr2[52] == ' ')
	sscanf(longstr2,"%2d %5ld %9lf %9lf %8lf %9lf %9lf %10lf %6ld \n",
                     &cardnumb,&satrec.satnum, &satrec.inclo,
                     &satrec.nodeo,&satrec.ecco, &satrec.argpo, 
                     &satrec.mo, &satrec.no, &revnum );
else
	sscanf(longstr2,"%2d %5ld %9lf %9lf %8lf %9lf %9lf %11lf %6ld \n",
                     &cardnumb,&satrec.satnum, &satrec.inclo,
                     &satrec.nodeo,&satrec.ecco, &satrec.argpo, 
                     &satrec.mo, &satrec.no, &revnum );
\end{lstlisting}


\section{Иммитатор радиоизмерений}

\begin{wrapfigure}[25]{l}{0.4\textwidth}
	\begin{center}
		\begin{tikzpicture}[node distance = 0.5cm, auto, align=center]
% Узлы
\node [start]                     (init)          {Начало\\ итерации};
\node [block, below of=init]      (obtain)        {Получение \\ координат КО \\ в текущий \\ момент времени};
\node [decision, below of=obtain, node distance=3.5cm] (view)          {Виден ли КО РЛС?};
\node [block, below of=view]      (calculate)     {Рассчет \\ наблюдаемых \\ параметров и т.д.};
%изменить тип блока записи
\node [block, below of=calculate, node distance=2.5cm] (write)         {Запись \\ результатов};
\node [end, below of=write, node distance=3cm]     (end)           {Конец\\ итерации};

% Рёбра
\path [line, dotted] (init) -- node {} (obtain);
\path [line] (obtain)       -- node {} (view);
\path [line] (view)         -- node {Да} (calculate);
\path [line] (calculate)    -- node {} (write);
\path [line, dotted] (write)-- node {} (end);
\draw [line] (view.east)    -- ++(1.5,0) 
                            -- ++(0,-8.5) 
                            -- ++(0,0) --                
node [xshift=1.8cm,yshift=4.7cm, text width=2.5cm]
{Нет} node {} (end.east);

%\path [line] (view) -- node {} (read);
\end{tikzpicture}
	\end{center}
	\caption{Граф потока управления иммитатора} \label{fig:control-flow-scheme}
\end{wrapfigure}
Иммитатор разделён на следующие подмодули
реализующие абстрактные модели (см. рисунок \ref{common-immitation-scheme}):
\begin{itemize}
	\item OrbitCalculator, модуль расчёта орбит спутников -- реализация модели движения КО.
	\item NeQuick, модуль расчёта влияния ионосферы -- реализация модели влияния ионосферы.
	\item RadarStation, модуль отвечающий за логику самой РЛС и проводимые её измерения -- реализация модели выбора
			КО и модели радиолокационных измерений, здесь же будет вноситься случайнаz ошибка измерений.
\end{itemize}

Так же в иммитатор входит \textit{ImmitationDriver} управляющий модуль, запускает и управляет циклами имитации,
направляет и именует потоки данных, может взаимодействовать с пользователем или внешними модулями и т.д

Определив все модули программы, можно определить то, как будет выглядеть один цикл вычислений, т.е. проведение
одного измерения радиолокационной станцией в определённый момент времени. Блок-схемы потока управления и потока
данных представлены на рисунке \ref{fig:control-flow-scheme} и \ref{fig:data-flow-scheme}.
\begin{figure}[H]
	\centering
	\begin{tikzpicture}[node distance = 1cm, auto, align=center]
% Узлы
\node [input]                                               (init)              {Начало итерации};
\node [block, below of=init]                                (TLEReader)         {TLEReader};
\node [block, below of=TLEReader]                           (AstroUtils)        {AstroUtils};
\node [block, below of=AstroUtils]                          (OrbitCalculator)   {OrbitCalculator};
\node [block, right of=TLEReader, node distance = 5cm]      (NeQuick)           {NeQuick};
\node [block, right of=AstroUtils, node distance = 5cm]     (RadarStation)      {RadarStation};
\node [block, right of=OrbitCalculator, node distance = 5cm](out)               {Выход};
\node [input, below of=out]                                 (end)               {Конец итерации};

% Рёбра
\path [line, dotted] (init)         -- node {} (TLEReader);
\path [line] (TLEReader)            -- node {} (AstroUtils);
\path [line] (AstroUtils)           -- node {} (OrbitCalculator);
\draw [line] (OrbitCalculator.east) -- ++(0.5,0) 
                                    -- ++(0,6) 
                                    -- ++(0,0) --                
node  {} node {} (NeQuick.west);
\path [line] (NeQuick)              -- node {} (RadarStation);
\path [line] (RadarStation)         -- node {} (out);
\path [line, dotted] (out)          -- node {} (end);

%\path [line] (view) -- node {} (read);
\end{tikzpicture}
	\caption{Граф потока данных иммитатора} \label{fig:data-flow-scheme}
\end{figure}

\textit{ImmitationDriver} занимается управлением циклами и слежением за тем, чтобы во время каждого цикла модулям
были доступны соответствующие этому циклу данные. \textit{TLEReader} в практической реализации не участвует в каждом 
цикле -- \textit{OrbitCalculator} обращается к нему только на первой итерации, для получения списка спутников 
и их параметров.

Как уже говорилось ранее, основные объекты, участвующие в имитации, хорошо и удобно объявляются в виде классов. В
модуле иммитации определены следующие классы:
\begin{mydef} \label{typ:GeoPoint}
Тип \textbf{GeoPoint} -- Класс для представления координат любых объектов в географической системе координат. 
Имеет методы \textit{GetLatitude}, \textit{GetLongitude}, \textit{GetAltitude}, \textit{DistanceTo} для получения соответственно широты, долготы, высоты и расстояния до любой другой заданной точки.
\end{mydef}
\begin{mydef} \label{typ:OrbitPoint}
Тип \textbf{OrbitPoint} -- Класс для представления координат любых объектов находящихся на орбите Земли 
в прямоугольной не вращающейся вместе с Землёй системе координат. Является наследником класса GeoPoint 
(см. Определение \ref{typ:GeoPoint})
Имеет методы \textit{GetIntertialX}, \textit{GetIntertialY}, \textit{GetIntertialZ} для получения 
координат x, y и z. А также метод \textit{GetTime} для получения момента времени в который спутник находился в 
указанных координатах.
\end{mydef}
\begin{mydef} \label{typ:RadarStation}
Тип \textbf{RadarStation} -- Класс для представления координат РЛС и её параметров. Является наследником класса 
GeoPoint (см. Определение \ref{typ:GeoPoint})
Имеет методы \textit{GetWorkFrequency} для получения рабочей частоты радара, \textit{IsInSigh} -- для получения
информации о том, попадает, переданная в качестве аргумента метода, точка в зону обнаружения РЛС (т.е. видит ли РЛС
эту точку), \textit{ObservedDistanceTo} для вычисления измеряемого расстояния до точки (т.е. с учётом ионосферы)
координат x, y и z. А также методы \textit{ZenithAngleTo} и \textit{AzimuthAngleTo} для получения углов наблюдения 
точки с учётом влияния ионосферы.
\end{mydef}

Вызов любого из методов: \textit{ObservedDistanceTo}, \textit{ZenithAngleTo} или \textit{AzimuthAngleTo}, фактически
приводит к запуску одной итерации цикла иммитации (см. рисунок \ref{fig:control-flow-scheme}).

Кроме этого в модуле есть классы для работы с орбитами, которые могут содердать в себе множество точек типа 
\textit{OrbitPoint} (\ref{typ:OrbitPoint}):
\begin{mydef} \label{typ:Orbit}
Тип \textbf{Orbit} -- Класс для представления орбиты спутника и её параметров. Позволяет вычислить и сохранить
местоположения спутника в указанные моменты времени. Вызов любого из его методов -- \textit{GetTrajectoryPoint} 
или \textit{GetTrajectoryPoints} запустит иммитацию по модели SDP/SGP (см. параграф \ref{sbs:space-object-model}).
\end{mydef}

Для сохранения результатов иммитации создан специальный класс \textit{SightReport}
\begin{mydef} \label{typ:SightReport}
Тип \textbf{SightReport} -- Класс представляющий из себя отчёт о наблюдении одной РЛС.в один момент времени
Сохраняет координаты наблюдения, время и параметры наблюдавшей РЛС, а так же истинные значения всех величин.
\end{mydef}

Из определений видно, что \textit{RadarStation} и \textit{OrbitPoint} наследуются от GeoPoint, таким образом, удаётся
добиться простоты в расчётах расстояния и координат для них. Для точек орбит функции GetLatitude, GetLongitude будут 
возвращать координаты подспутниковой точки.

Схем классов слегка отличаются от тех, что были составлены для модулей иммитатора, главным образом в целях
оптимизации производительности и читаемости кода.

Размер случайной погрешности будет задаваться через сигму и рассчитываться при помощи встроенной в С++
математической библиотеки (см. листинг \ref{lst:RandomizeValue}).
\begin{lstlisting}[language=C++, basicstyle=\fontsize{10}{9}\ttfamily,
	caption={Добавление случайной велечины к измеренной},label={lst:RandomizeValue}]
double astroutils::RandomizeValue(double expected, double sigma)
{
    assert(sigma < 1);
    double normalized_sigma = fabs(expected * sigma);
    std::normal_distribution<double> distribution(expected, normalized_sigma);

    return distribution(astroutils::generator);
}
\end{lstlisting}

Полностью описать работу иммитатора можно блок-схемой изображённой на рисунке \ref{fig:immitator-full-scheme}
\begin{figure}[H]
	\centering
	\input{graphs/immitator-full-scheme}
	\caption{Полная схема работы иммитатора} \label{fig:immitator-full-scheme}
\end{figure}
\newpage

\section{Восстановитель орбит}

Для реализации метода градиентного спуска была использована библиотека GSL\footnote{Scientific Library}. Для 
сравнения качества восстановления орбиты с учётом и без учёта ионосферы необходимо реализовать два метода для 
каждого из них соответсвенно, методы различаются лишь частью связанной с получением параметров т.к. для одного из
них отсутсвует ионосферный параметр -- солнечная активность (см
\begin{lstlisting}[language=C++, basicstyle=\fontsize{10}{9}\ttfamily,
	caption={Добавление случайной велечины к измеренной},label={lst:RandomizeValue}]
int orbitresolver::CalculationFunctionWithFlux(const gsl_vector * x, void *data,
                                gsl_vector * f)
{
    restore_function_parameters *parameters =
            (restore_function_parameters *) data;
    const size_t result_size = parameters->objects_observations->size();
    time_t start_time =
            parameters->objects_observations->begin()->GetObservationTime();
    time_t end_time =
            parameters->objects_observations->back().GetObservationTime();
    time_t epoch =
            parameters->real_orbit->GetEpochTime();
    std::list<SightObject> *objects_observations =
            parameters->objects_observations;
    RadarStation *radar = parameters->observer;

    double drag_coefficient = gsl_vector_get(x, 0);
    double inclination_angle = gsl_vector_get(x, 1);
    double ascending_node = gsl_vector_get(x, 2);
    double eccentricity = gsl_vector_get(x, 3);
    double apsis_argument = gsl_vector_get(x, 4);
    double mean_anomaly = gsl_vector_get(x, 5);
    double mean_motion = gsl_vector_get(x, 6);

    Orbit orbit(0, *parameters->real_orbit,
                drag_coefficient, inclination_angle, ascending_node,
                eccentricity, apsis_argument, mean_anomaly, mean_motion);
    orbit.SetEpoch(parameters->real_orbit->GetEpochTime());

    radar->SetSigma(0);
//    radar->SetLocalFlux(gsl_vector_get(x, 7));
    radar->SetLocalFlux(parameters->observer->GetLocalFlux());

    ImitationDriver driver;
    driver.addRadarStation(*radar);
    driver.addSatellite(orbit);

    auto simulation_report = driver.RunImitation(start_time, end_time);
    std::list<SightObject> result_objects;
    for(auto it = simulation_report.begin(); it != simulation_report.end();
        ++it)
    {
        auto observation_report = it->second;
        for(auto observation_iterator = observation_report.begin();
            observation_iterator != observation_report.end();
            ++observation_iterator) {
            if(observation_iterator->GetObservedObjects().size() < 1)
                result_objects.push_back(SightObject(0, 0, 0, 0, 0));
            else
            {
                auto observe = *observation_iterator;
                result_objects.push_back(observe.GetObservedObjects()[0]);
            }
        }
    }

    size_t i = 0;
    for (auto it = result_objects.begin(),
                 it2 = objects_observations->begin();
         it != result_objects.end() && it2 != objects_observations->end()
         && i < result_size; ++it, ++it2, ++i)
    {
        double delta = pow(it->GetDistanceTo()
                       - it2->GetDistanceTo(), 2);
        //std::cout << "Delta : " << delta << std::endl;
        gsl_vector_set(f, i, delta);
    }

    return GSL_SUCCESS;
}
\end{lstlisting}